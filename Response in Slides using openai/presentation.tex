
\documentclass{beamer}
\usepackage[utf8]{inputenc}
\usetheme{Madrid}

\title{Force in Physics}
\author{Your Name}
\date{\today}

\begin{document}

\frame{\titlepage}

\begin{frame}{Introduction}
    \begin{itemize}
        \item Force is a fundamental concept in physics that describes the interaction between objects, causing a change in their motion.
        \item According to Newton's second law of motion, when a net force acts on an object, it produces acceleration in the object in the direction of the net force.
        \item The magnitude of acceleration is directly proportional to the net force and inversely proportional to the mass of the object.
    \end{itemize}
\end{frame}

\begin{frame}{Newton's Second Law}
    \begin{itemize}
        \item Mathematically, force (F) is equal to mass (m) multiplied by acceleration (a), represented as $F = ma$.
        \item The SI unit of force is the newton (N), defined as the force required to produce an acceleration of 1 m/s$^2$ in a body of mass 1 kg.
    \end{itemize}
\end{frame}

\begin{frame}{Effects of Forces}
    \begin{itemize}
        \item Forces can move, stop, or change the direction of motion of an object.
        \item Forces can also change the shape or size of the object they act upon.
    \end{itemize}
\end{frame}

\begin{frame}{Newton's Third Law}
    \begin{itemize}
        \item Forces can be classified as action-reaction pairs, following Newton's third law of motion.
        \item Newton's third law states that for every action, there is an equal and opposite reaction.
    \end{itemize}
\end{frame}

\end{document}
